Il paradigma dell'Internet of Things ha aperto la strada verso un mondo in cui i nostri oggetti di uso comune saranno interconnessi e interagiranno con l'ambiente esterno per raccogliere informazione e automatizzare task, come ad esempio l'accensione di luci in una casa, l'azionamento di sistemi di irrigazione e così via. Questa visione implica una serie di feature, come autenticazione, privacy dei dati, sicurezza, robustezza agli attacchi, sviluppo agevole delle applicazioni e self-maintenance\cite{Fernandez-Carames2018}.

Queste caratteristiche possono essere implementate attraverso l'uso della \textit{Blockchain}, una tecnologia nata insieme alla famosa criptovaluta \textit{Bitcoin}. 

Pubblicato nel 2008, il white paper\cite{Nakamoto2008} che ne descrive il funzionamento porta il nome di Satoshi Nakamoto. Tuttora non è chiaro che egli sia, non è neanche certo che si tratti di una persona sola. \\

\\ *** ALTRE PUPPETTE **** \\

\\
%% PIÙ PUPPETTE %%

Questo lavoro di tesi si concentra sull'integrazione della blockchain in un sistema basato sulla Social Internet of Things, il GRIDS (Generic Resilient Identity Services) System, il quale permette di migliorare le operazioni di mapping ID-to-locator attraverso la navigazione del grafo SIoT creato dalle relazioni che si stabiliscono tra gli oggetti.

Il capitolo \ref{c:tec} parla dello stato dell'arte: per quanto riguarda l'Internet of Things, verrà fatta una veloce panoramica sulla sua evoluzione, con una particolare attenzione sulla Social IoT, citando inoltre alcune piattaforme di virtualizzazione. Verranno presentate anche le caratteristiche principali della blockchain, portando come esempio la rete Bitcoin con il suo algoritmo per il raggiungimento del consenso distribuito \textit{Proof of Work}. 
Saranno introdotti anche elementi di crittografia, fondamentale per poter capire appieno il funzionamento della blockchain.

Verrà descritto il GRIDS System nel capitolo \ref{c:grids}, con particolare attenzione sull'architettura e sull'implementazione in Java....

Il capitolo \ref{c:integr} costituisce la parte centrale del lavoro: saranno coomentati i passi eseguiti per l'integrazione della blockchain nel sistema, dalla scelta della blockchain stessa, all'interfaccia grafica per l'interazione dell'utente con il GRIDS System, con particolare attenzione rivolta alle scelte implementative. Per contestualizzare il lavoro, sono anche descritti i principali casi d'uso.

Il capitolo successivo, invece, si concentra sulla valutazione delle performance, mettendo a confronto i risultati ottenuti dai vari casi d'uso.

Per concludere, nel capitolo \ref{c:end} verranno suggeriti dei potenziali sviluppi futuri per migliorare ulteriormente il sistema.