Il paradigma dell'Internet of Things ha aperto la strada verso un mondo in cui i nostri oggetti di uso comune saranno interconnessi e interagiranno con l'ambiente esterno per raccogliere informazione e automatizzare alcuni task. Questa visione implica una serie di feature, come autenticazione, privacy dei dati, sicurezza, robustezza agli attacchi, sviluppo agevole delle applicazioni e self-maintenance.\cite{Fernandez-Carames2018}

Queste caratteristiche possono essere implementate attraverso l'uso della \textit{Blockchain}, una tecnologia nata insieme alla famosa criptovaluta \textit{Bitcoin}. 



Questo lavoro di tesi si concentrerà sull'integrazione della blockchain in un sistema basato sulla Social Internet of Things, il GRIDS (Generic Resilient Identity Services) System, il quale permette di migliorare le operazioni di mapping ID-to-locator attraverso la navigazione del grafo SIoT creato dalle relazioni che si stabiliscono tra gli oggetti.

Il capitolo \ref{c:tec} parla dello stato dell'arte: per quanto riguarda l'Internet of Things, verrà fatta una veloce panoramica sulla sua evoluzione, con una particolare attenzione sul Social IoT, citando inoltre alcune piattaforme di virtualizzazione. Verranno presentate anche le caratteristiche principali della blockchain, portando come esempio la rete Bitcoin.... 

Ci sono anche elementi di crittografia, fondamentale per poter capire appieno il funzionamento della blockchain.....

Verrà introdotto il GRIDS System nel capitolo \ref{c:grids}, con particolare attenzione sull'architettura e sull'implementazione in Java....

Il capitolo \ref{c:integr} costituisce la parte centrale del lavoro: saranno illustrati i passi seguiti per l'integrazione della blockchain nel sistema, dalla scelta della blockchain stessa all'interfaccia grafica per l'interazione dell'utente con il sistema...

Il capitolo successivo si concentra sulla valutazione delle performance ....

Per concludere, verranno suggeriti dei potenziali sviluppi futuri che possano migliorare ulteriormente il sistema......