In questo lavoro di tesi è stato presentato un sistema che realizza il mapping... ed è stato integrato con la blockchain....

dopo una valutazione sulle tecnologie utilizzabili per l'integrazione della blockchain nel sistema, si è passati all'implementazione, grazie alla libreria Bitcoinj per sviluppare client completi di wallet e funzioni per effettuare le transazioni....

Il sistema completo di blockchain è risultato essere ...

soprattutto con il meccanismo di ricarica si è riusciti a bypassare i tempi di conferma delle transazioni....

cose fighe...


Qui di seguito sono proposti dei possibili sviluppi futuri per il sistema: 
\begin{itemize}
    \item deployare il sistema nella rete Bitcoin ufficiale: questo implica l'acquisto di criptovaluta oppure il mining di bitcoin con macchine sufficientemente potenti per supportare l'algoritmo di Proof of Work, C'È DA UNIRSI A UN POOL DI MINER, OVVIAMENTE NON POSSIAMO ESSERE MINER DA SOLI, NO SE PUEDE;
    \item realizzare una blockchain from scratch, PERÒ CI VOGLIONO TANTE RISORSE;
    \item usare altre blockchain (Ethereum, Ripple);
    \item aggiungere gli Smart Contracts\footnote{https://en.bitcoin.it/wiki/Contract}, cioè INSERIRE DESCRIZIONE SUGLI SMART CONTRACTS PIÙ PAPER, per automatizzare i meccanismi di retrieval dei dati.
\end{itemize}
